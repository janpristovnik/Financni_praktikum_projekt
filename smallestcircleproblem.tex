\documentclass[a4paper]{article}
\usepackage[slovene]{babel}
\usepackage[utf8]{inputenc}
\usepackage[T1]{fontenc}
\usepackage{graphicx}
\usepackage{marvosym}
\usepackage{amssymb,amsmath}
\usepackage{algorithm}
\usepackage{algorithmic}


\newtheorem{exmp}{Primer}

\title{Problem najmanjšega kroga}
\author{Jan Prostovnik in Jan Lampič\\ Finančni praktikum \\ Finančna matematika, Fakulteta za matematiko in fiziko}
\date{20.12.2017}


\begin{document}
\title{
  Problem najmanjšega kroga}

\author{Jan Pristovnik in Jan Lampič}

\maketitle

\pagebreak

\section{Predstavitev problema}

Problem najmanjšega kroga ali najmanjšega pokrivnega kroga je matematični problem izračunavanja najmanjšega kroga, ki vsebuje vse določene množice
točk v evklidski ravnini. Ustrezni problem v $n$-dimenzionalnem prostoru, najmanjši problem z omejevalno sfero, je izračun najmanjše $n$-dimenzionalne sfere, ki
vsebuje vse določene množice točk. Problem najmanjšega kroga je prvotno
predlagal angleški matematik James Joseph Sylvester leta 1857.
\\
\\
Problem najmanjšega kroga v ravnini je primer težave z lokacijo objekta
(problem z enim središčem), v katerem je treba izbrati lokacijo novega
objekta, ki bo zagotavljal storitve vsem v naprej izbranim strankam. Nov
objekt mora biti postavljen tako, da bo čim bolj zmanjšal najdaljšo
razdaljo, ki jo mora katera koli stranka prepotovati, da doseže novi
objekt.
\newline Spodaj sta prikazana dva primera, prvi prikazuje problem najmanjšega kroga, drugi primer pa prikazuje kako lahko s pomočjo problema najmanjšega kroga rešujemo probleme iz realnega življenja. 

\begin{exmp}
Glede na dano množico točk v ravnini, izračunaj najmanjši krog, ki jo vsebuje.

\begin{figure}[ht]
\includegraphics [scale = 0.4]{krog}
\end{figure}
 
\end{exmp}

\begin{exmp}
Na nekem izoliranem območju stoji niz hiš. Kam moramo posatviti bolnišnico, da bomo minimizirali najdaljšo razdaljo vsake posamezne hiše do bolnišnice?\\
Kje naj postavimo anteno, da bo imelo čim več lokacij sprejem?

\begin{figure}[ht]
\includegraphics [scale = 0.4]{facility}

\end{figure}
\end{exmp}

\newpage
\section{Karakterizacija problema}

Večina geometrijskih pristopov za problem išče točke, ki ležijo na meji
najmanjšega kroga in temeljijo na naslednjih preprostih dejstvih:
\begin{itemize}
\item Najmanjši krog, ki pokriva vse izbrane točke je edinstven
\item Najmanjši krog, ki pokriva vse točke iz množice $\mathcal{P}$ lahko določimo z
največ tremi točkami, ki ležijo na robu in so vsebovane v $\mathcal{P}$. Če je
krog določen samo z dvema točkama, mora razdalja med točkama
predstavljati premer minimalnega kroga. Če ga sestavljajo tri
točke, potem trikotnik sestavljen iz teh treh točk, ne vsebuje
topih kotov.
\end{itemize}

Kot je pokazal Nimrod Megiddo, je problem iskanja najmanjšega kroga, ki
pokriva vse izbrane točke, rešljiv v linearnem času. Enako velja tudi za
iskanje najmanjše $n$-sfere, ki vsebuje vse množice točk v $n$-dimenzionalnem
prostoru.

\section{Algoritem Welzla}

V najini nalogi se bova osredotočila na algoritem \textbf{Emo Welzla}. Algoritem temelji na linearno programskem algoritmu Raimunda Seidela. Izbrala sva ga zaradi njegove enostavnosti in časovne zahtevnosti, ki je linearna.

\subsection{Ideja algoritma in psevdo koda}

Kako skonstruiramo najmanjši krog $D_n$  (\textit{angl. disk}), ki vsebuje $n$-točk ?
\newline Če že poznamo najmanjši krog $D_{n-1}$, ki pokriva  prvih $n-1$ točk ($p_1 ,…,p_{n-1}$), potem sta za $n$-to točko možna dva primera.

\begin{enumerate}
\item Če točka $p_n$ leži znotraj kroga $D_{n-1}$, se nič ne spremeni –  krog $D_{n-1}$ za točke $p_1 ,…,p_{n-1}$ je enak kot $D_n$ za točke $p_1,...,p_n$.
\item  Če $p_n$ ne leži znotraj kroga $D_{n-1}$, je treba izračunati nov krog. Vemo pa, da mora $n$-ta točka ležati na robu kroga. Torej moramo izračunati najmanjši krog, ki vsebuje točke $p_1 ,…,p_{n-1}$ in ima $p_n$ na robu.
\end{enumerate}


Ta lastnost skupaj z naslednjimi zahtevki nam omogoča, da najmanjši krog izračunamo na sledeč iterativen način.
Naj bo $\mathcal{P}$ neprazna množica $n$ točk v ravnini in $p$ točka v $\mathcal{P}$. $\mathcal{P}$ pa naj bo množica  robnih točk. 
Potem velja:

\begin{itemize}
\item Če obstaja krog, ki vsebuje $\mathcal{P}$ in ima $\mathcal{R}$ na robu, potem je dobro definiran in enoličen.
\item Če $p$ ne leži v krogu $D(\mathcal{P} - \{p\},\mathcal{R})$, potem p leži na robu kroga $D(\mathcal{P},\mathcal{R})$, pod pogojem, da obstaja. To pomeni, da je $D(\mathcal{P},\mathcal{R}) = D(\mathcal{P} - \{p\}, \mathcal{R} \cup \{p\})$.
\item Če $D(\mathcal{P},\mathcal{R})$ obstaja, potem obstaja množica $\mathcal{S}$, ki vsebuje $max\{0, 3 - |\mathcal{R}|\}$ točk v $\mathcal{P}$ tako, da je $D(\mathcal{P},\mathcal{R}) = (\mathcal{S},\mathcal{R})$. To pomeni, da je najmanjši krog, ki vsebuje $\mathcal{P}$ določen z največ 3 točkami iz $\mathcal{P}$, ki ležijo na robu kroga.
\end{itemize}

S temi lastnostmi lahko algoritem implementiramo na rekurziven način.\\

\begin{algorithm}[h]
\caption {Welzl}
\begin{algorithmic} 
\REQUIRE Končni množici $\mathcal{P}$ in $\mathcal{R}$ točk v ravnini
\ENSURE Najmanjši krog krog $D(\mathcal{P},\mathcal{R})$, ki vsebuje množico točk $\mathcal{P}$ in ima $\mathcal{R}$ na robu
\IF{$\mathcal{P}$ je prazna ali $|\mathcal{R}| \geq 3$ }
\RETURN Krog $D$, ki ga točke določajo.
\ELSE
\STATE \textbf{Choose} $p$ iz $P$ naključno;
\STATE $D := Welzl(\mathcal{P}-\{p\},\mathcal{R})$;
\IF{$p$ je v $D$}
\RETURN $D$
\ENDIF
\ENDIF
\RETURN $Welzl(\mathcal{P} - \{p\}, \mathcal{R} \cup \{p\})$
\end{algorithmic}
\end{algorithm}




\section{Načrt dela}

V najinem nadaljnem delu bova Welzlov algoritem zapisala v programskem jeziku \texttt{Python}. Algoritem bova aplicirala na različnih množicah točk. Vpeljimo sedaj nekaj oznak, da bova lahko razložila cilje najinega projektnega dela. Naj bo, za množico točk $\mathcal{A}$,$ r(\mathcal{A})$ radij najmanjšega kroga, ki vsebuje $\mathcal{A}$.  Naj bo $\mathcal{C}$ konveksno območje v ravnini in $P_n$ vzorec $n$-tih naključnih točk iz $\mathcal{C}$. Zanimalo naju bo :
\begin{itemize} 
\item Kako hitro $r(P_n)$ konvergira k $r(\mathcal{A})$?
\item V kolikšni meri je hitrost zgornje konvergence odvisna od oblike območja $\mathcal{C}$?
\item Katera oblika območja $\mathcal{C}$ ima najpočasnejšo konvergenco? 
\end{itemize}

Če projekt ne bo dovolj obsežen se bova posvetila še drugim algoritmom, ki rešujejo problem najmanjšega kroga in jih primerjala z Welzlovim.



\end{document}