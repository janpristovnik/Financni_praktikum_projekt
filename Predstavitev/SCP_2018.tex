\documentclass[a4paper]{article}
\usepackage[slovene]{babel}
\usepackage[utf8]{inputenc}
\usepackage[T1]{fontenc}
\usepackage{graphicx}
\usepackage{marvosym}
\usepackage{amssymb,amsmath}
\usepackage{algorithm}
\usepackage{algorithmic}
\usepackage{enumerate}


\newtheorem{exmp}{Primer}
\newtheorem{lema}{Lema}
\newtheorem{izrek}{Izrek}

\title{Problem najmanjšega kroga}
\author{Jan Prostovnik in Jan Lampič\\ Finančni praktikum \\ Finančna matematika, Fakulteta za matematiko in fiziko}
\date{5.1.2018}


\begin{document}
\title{ Problem najmanjšega kroga}

\author{Jan Pristovnik in Jan Lampič}

\maketitle

\pagebreak

\section{Predstavitev problema}

Problem najmanjšega kroga ali najmanjšega pokrivnega kroga je matematični probelm izračunavanja najmanjšega kroga, ki vsebuje vse določene množice točk v evklidski ravnini. Če želimo problem rešiti glede na dane točke $P = \{p_1,p_2,..,p_n\}$ v evklidski ravnini,  moramo poiskati središče in polmer kroga, ki bo vseboval vse točke iz množice $P$. 
Ustrezni problem v $n$-dimenzionalnem prostoru, najmanjši problem z omejevalno sfero, je izračun najmanjše $n$-dimenzionalne sfere, ki vsebuje vse določene množice točk. Problem najmanjšega kroga je prvotno predlagal angleški matematik Joseph Sylvester leta 1857.\\ 
Problem najmanjšega kroga se pojavlja na različnih področjih uporabe, kot so:
\begin{itemize}
\item Primer težave z lokacijo objekta, v katerem je treba izbrati lokacijo novega objekta, ki bo zagotavljal storitve vsem v naprej izbranim strankam. Nov objekt mora biti postavljen tako, da bo čim bolj zmanjšal najdaljšo razdaljo, ki jo mora katera koli stranka prepotovati, da doseže novi objekt.
\item Za reševanje problemov v okoljski znanosti (načrtovanje in optimizacija topil), prepoznavanje vzorcev (iskanje referenčnih točk), biologija (analiza beljakovin), politična znanost (analiza stranskih spektrov), strojništvo (optimizacija stresa) in računalniška grafika (sledenje žarkov, izločanje) itd.
\end{itemize}
Najenostavnejši algoritem bi bil, preveriti vse kroge definirane z dvema in tremi točkami, ter preveriti ali vsebuje vse ostale točke in nakoncu izbrati najmanjšega med ustreznimi.  Če bi imeli prvotno $n$ točk, bi bilo takih krogov $O(n^3)$. Za vsak krog bi potrebovali $O(n)$, da bi preverili ali vsebuje vse ostale točke, torej skupna časovna zahtevnost bi bila $O(n^4)$. Tak algoritem se je prvič pojavil že okoli leta 1869, od takrat se je zgodilo veliko izboljšav. Danes poznamo veliko praktično uporabnih algoritmov. Teoretično najbolj izpopolnjeni algoritmi imajo časovno zahtevnost $O(n)$.  

\section{Karakterizacija problema}

Večina geometrijskih pristopov za problem išče točke, ki ležijo na meji
najmanjšega kroga in temeljijo na naslednjih preprostih dejstvih:
\begin{itemize}
\item Najmanjši krog, ki pokriva vse izbrane točke je enoličen.
\item Najmanjši krog, ki pokriva vse točke iz množice $P$ lahko določimo z
največ tremi točkami, ki ležijo na robu in so vsebovane v $P$. Če je
krog določen samo z dvema točkama, mora razdalja med točkama
predstavljati premer minimalnega kroga. Če ga sestavljajo tri
točke, potem trikotnik sestavljen iz teh treh točk, ne vsebuje
topih kotov.
\end{itemize}

Kot je pokazal Nimrod Megiddo, je problem iskanja najmanjšega kroga, ki
pokriva vse izbrane točke, rešljiv v linearnem času. Enako velja tudi za
iskanje najmanjše $n$-sfere, ki vsebuje vse množice točk v $n$-dimenzionalnem
prostoru.


\section{Algoritem Welzla}

V najini nalogi sva se osredotočila na algoritem \textbf{Emo Welzla}. Algoritem temelji na linearno programskem algoritmu Raimunda Seidela. Izbrala sva ga zaradi njegove enostavnosti in časovne zahtevnosti, ki je pričakovano linearna.

\subsection{Ideja algoritma}

Kako skonstruiramo najmanjši krog $D_n$  (\textit{angl. disk}), ki vsebuje $n$-točk ?
\newline Če že poznamo najmanjši krog $D_{n-1}$, ki pokriva  prvih $n-1$ točk ($p_1 ,…,p_{n-1}$), potem sta za $n$-to točko možna dva primera.

\begin{enumerate}
\item Če točka $p_n$ leži znotraj kroga $D_{n-1}$, se nič ne spremeni –  krog $D_{n-1}$ za točke $p_1 ,…,p_{n-1}$ je enak kot $D_n$ za točke $p_1,...,p_n$.
\item  Če $p_n$ ne leži znotraj kroga $D_{n-1}$, je treba izračunati nov krog. Vemo pa, da mora $n$-ta točka ležati na robu kroga. Torej moramo izračunati najmanjši krog, ki vsebuje točke $p_1 ,…,p_{n-1}$ in ima $p_n$ na robu.
\end{enumerate}


Ta lastnost skupaj z naslednjimi zahtevki nam omogoča, da najmanjši krog izračunamo na sledeč iterativen način.
Naj bo $P$ neprazna množica $n$ točk v ravnini in $p$ točka v $P$. $R$ pa naj bo množica  robnih točk. 
Potem velja:

\begin{itemize}
\item Če obstaja krog, ki vsebuje $P$ in ima $R$ na robu, potem je dobro definiran in enoličen.
\item Če $p$ ne leži v krogu $D(P - \{p\},R)$, potem p leži na robu kroga $D(P,R)$, pod pogojem, da obstaja. To pomeni, da je $D(P,R) = D(P - \{p\}, R \cup \{p\})$.
\item Če $D(P,R)$ obstaja, potem obstaja množica $S$, ki vsebuje $max\{0, 3 - |R|\}$ točk v $P$ tako, da je $D(P,R) = D(S,R)$. To pomeni, da je najmanjši krog, ki vsebuje $P$ določen z največ 3 točkami iz $P$, ki ležijo na robu kroga.
\end{itemize}


\subsection{Algoritem - koda}

Definirala sva funkcijo \texttt{Welzl}, ki za vhodne podatke sprejme množico točk v ravnini $P$, vrne pa najmanjši krog, ki vsebuje vse točke iz $P$. 
V prvem delu algoritma poskrbiva za pravilen tip podatkov, s funkcijo $random.shuffle$ pa poskrbiva za naključno izbiro točk.
Za množico točk $P$ izračunamo najmanjši krog postopoma. Začnemo s prazno množico, ki ji zaporedoma dodajamo točke. Za vsako točko preverimo, ali je vsebovana v do sedaj najmanjšem krogu, v tem primeru gremo k naslednji točki. Če točka $p$ ni vsebovana, iz karakterizacije problema vemo, da leži na robu kroga. Nov najmanjši krog izračunamo s pomočjo funkcije \texttt{krog1}.\\ 
Funkcija \texttt{krog1} sprejme do sedaj obravnavane točke in eno robno točko $p$. Vrne najmanjši krog, ki vsebuje do sedaj obravnavane točke s točko $p$ na robu. 
SLIKA\\
Najprej za začetni krog vzamemo kar točko $p$ ter ponovno oblikujemo najmanjši krog. Za vsako točko $q$ preverimo, če ni vsebovana v do sedaj izračunanem najmanjšem krogu. Če je izpolnjen prvi pogoj je nov najmanjši krog enolično določen z dvema točkama $p$ in $q$ - klic funkcije \texttt{premer($p$,$q$)}, ki vrne krog, ki ga določata točki $p$ in $q$, njuna razdalja pa je kar premer kroga. Sicer vemo, da je točka $q$ na robu in kličemo funkcijo \texttt{krog2}. V primeru, da je točka $q$ vsebovana v najmanjšem izračunanem krogu, se krog ne spremeni in nadaljujemo s for zanko.
Funkcija \texttt{krog2} sprejme do sedaj obravnavane točke in dve robni točki $p$ ter $q$. Vrne pa najmanjši krog, ki vsebuje vse obravnavane točke s točkama $p$ in $q$ na robu. Začetni krog nastavimo s funkcijo \texttt{premer($p,q$)}.  Sedaj obravnavamo samo točke, ki niso vsebovane v začetnem krogu.
Brez škode za splošnost lahko predpostavimo, da točki $p$ in $q$ ležita na vertikalni premici $\ell$. Za vse točke, ki so levo od premice $\ell$, poiščemo tisto točko $p_l$, ki s $p$ in $q$ določa krog s središčem najbolj levo od $\ell$. Postopek ponovimo za točke desno od $\ell$ in dobimo točko $p_d$. 

Če so vse obravnavane točke vsebovane v začetnem krogu, se krog ne spremeni. Sicer vrnemo manjšega izmed krogov \texttt{ocrtan\_krog($p,q,p_l$)} in \texttt{ocrtan\_krog($p,q,p_d$)}, kjer funkcija \texttt{ocrtan\_krog} vrne očrtan krog določen s tremi točkami.
Slika

\section{Eksperiment}

\subsection{Konstrukcija eksperimenta}

Naj bo, za množico točk $P$, $r(P)$ radij najmanjšega kroga, ki vsebuje $P$. Naj bo $C$ konveksno območje v ravnini in $P_n$ vzorec $n$-tih nakjučnih točk iz $C$. 
Pri projektu naju je zanimalo:
\begin{enumerate}
\item Kako hitro $r(P_n)$ konvergira k $r(P)$?
\item V kolikšni meri je hitrost zgornje konvergence odvisna od oblike območja $C$?
\item Katera oblika območja $C$ ima najpočasnejšo konvergenco?
\end{enumerate}
Glede na zgornje probleme sva eksperiment zasnovala na sledeč način.

>SLIKA

Izbrala sva si različna konveksna območja krog, elipso, kvadrat, pravokotnik in trikotnik. 
Za vsako območje sva preverila hitrost konvergence pri različnem številu točk ($10^2,10^3,10^4,10^5$).
Glavni eksperiment sva za vsako območje in izbrano število točk ponovila 100-krat. V vsakem poskusu sva s pomočjo funkcije \texttt{randomTocke} zgenerirala točke znotraj območja, na katerih sva izvedla \textbf{Welzlov} algoritem ter merila čas konvergence. Nato sva vzela povprečje 100-ih časov in ga shranila v slovar.\\
Opombe:
\newline Funkcija \texttt{randomTocke} za vhodne podatke prejme število točk $n$ ter obliko območja. Funkcija priredi najmanjši pravokotnik, ki vsebuje dano območje. Nato naključno generira točke znotraj pravokotnika in s pomočjo funkcije \texttt{je\_v\_Polygonu} preveri ali je točka vsebovana v iskanem območju.  Na koncu vrne $n$ naključnih točk znotraj danega območja. 
Za elipso in krog sva funkcijo \texttt{randomTocke} priredila.

\section{Rezultati}

Za začetni eksperiment sva primerjala vse oblike med seboj. 
»insert graf«
Isti eksperiment sva ponovila pri različnih velikostih likov. Opazila sva, da sama velikost nima vpliva na rezultate, odstopanja so prisotna zaradi premalega števila poskusov. 
Največje odstopanje je bilo v vseh poskusih opaziti pri krogu. Ker je elipsa imela veliko hitrejšo konvergenco, sva dodatno analizo naredila pri različnih parametrih za elipso v primerjavi s krogom. 
\newline Elipso sva naredila zelo ploščato(Elipsa3), »običajno«(Elipsa2) in še bolj podobno krogu(Elipsa1). Rezultati so prikazani v spodnjem grafu.
»insert graf«
Bolj kot je elipsa ploščata hitrejša je konvergenca, bolj kot je podobna krogu počasnejša je, takšne rezultate sva pričakovala zaradi rezultatov prvega eksperimenta.
Naknadno naju je zanimalo, če bi podobne rezultate dobila tudi z manipuliranjem pravokotnika v primerjavi s kvadratom. Ker se pravokotnik in kvadrat v prvem eksperimentu nista močno razlikovala, v naknadni analizi nisva pričakovala velikih odstopanj. Konvergence so prikazane v spodnjem grafu.
»insert graf«
Pravokotnik1 predstavlja najbolj ploščatega, Pravokotnik2 »običajnega« in Pravokotnik3 najbolj podobnega kvadratu.  Močno odstopanje dobimo le pri najbolj ploščatem pravokotniku, kar se ujema z rezultati različnih elips. 

\subsection{Povzetek }

S pridobljenimi rezultati sva prišla do sledeših ugotovitev:
\begin{itemize}
\item Hitrost konvergence Welzlovega algoritma je pričakovano linearna, kar je lepo vidno v rezultatih prikazanih na grafih.
\item Oblika območja predstavlja konstantni večkratnik pri linearni konvergenci.
\item Najpočasnejšo konvergenco izmed konveksnih območij ima krog, najboljšo pa ploščat pravokotnik. 

\end{itemize}

	»Razloge za hitro konvergenco pri ploščatem pravokotniku vidiva v tem, da ko enkrat dobimo točko na levem ali desnem robu pravokotnika, je najmanjši krog, ki vsebuje vse točke iz pravokotnika zelo blizu optimalne rešitve. 

\section{Viri}
\begin{itemize}
\item Sheng Xu, Robert M. Freund, Jie Sun, Solution Methodologies for the Smallest Enclosing Circle Problem (2003)
\item Emo Welzl, Smallest enclosing disks. \textit{New results and New Trends in Computer Science} (1991)  
\item <https://en.wikipedia.org/wiki/Smallest-circle\_problem>[online][5.1.2018]
\item <http://code.activestate.com/recipes/66527-finding-the-convex-hull-of-a-set-of-2d-points/>[online][5.1.2018]
\item  <http://www.cs.uu.nl/docs/vakken/ga/slides4b.pdf>[online][5.1.2018]
\item  <https://www.nayuki.io/page/smallest-enclosing-circle>[online][5.1.2018]
\end{itemize}


\end{document}